\documentclass[12pt]{article}
\usepackage{graphicx}
%\documentclass[journal,12pt,twocolumn]{IEEEtran}
\usepackage[english]{babel}
\usepackage{amsmath}   % for having text in math mode
\usepackage{ragged2e} % For text alignment

\begin{document}

\begin{figure}[h]
    \centering
    \begin{minipage}{0.4\textwidth}
        \includegraphics[width=\linewidth]{logo.jpg} % Repl>
    \end{minipage}
    \begin{minipage}{0.55\textwidth}
        \flushright % Aligns text to the right
        \textbf{Name:N.Bhanu Thahera} \\[5pt]
         \textbf{ID:COMETfwc005} \\
          \textbf{Date:27-03-2025} \\
    \end{minipage}
\end{figure}


\begin{center}
\textbf\large{GATE \\ EE2010}
\end{center}



\begin{enumerate}
\item{\bfseries Q)}
 Which of the following circuits is a realization of the function.\[
F = \overline{X} \overline{Y} + YZ
\]

\begin{figure}[!h]
\centering
  \includegraphics[width=\columnwidth]{Img.jpg}
\end{figure}

{\bfseries Answer:}
(A)circuit Analysis\\
\begin{enumerate}
\item The first NAND gate takes inputs X as x' and another AND gate takes x,x' and gives as \(\overline{x}\) .\\

\item The second AND gate takes inputs y,z = yz.\\

\item The output F is obtained by AND-ing the outputs of both AND gates.\\

\item Boolean Expression for (A):\\
\[
F = \overline{x}yz
\]
\end{enumerate}
(B)circuit Analysis\\
\begin{enumerate}
\item The first AND gate takes inputs X as x .

\item The second AND gate takes inputs y,z = yz.

\item The output F is obtained by OR-ing the outputs of both AND gates.

Boolean Expression for (A):
\(
F =X+YZ
\)\\
\end{enumerate}
(C)circuit Analysis\\
\begin{enumerate}
\item The first NAND gate takes inputs X as x' and another AND gate takes x,x' and gives as\(\overline{x}.\)

\item The second NAND gate takes inputs y,z = \(\overline{y}\overline{z}.\)

\item The output F is obtained by AND-ing the outputs of both AND gates.

\item Boolean Expression for (A):\\
\[
F = \overline{x}yz
\]
\end{enumerate}
(D)circuit Analysis\\
\begin{enumerate}
\item The first NAND gate takes input X as x' and another NAND gate takes input y as y'  and both inputs are given to NAND gate and gives \(\overline{x}\overline{y}.\)

\item The y input and z input is taken by AND gate and gives  yz.

\item The output F is obtained by OR-ing the outputs of both AND gates.

Boolean Expression for (A):\\
\[
F = \overline{x}\overline{y}+yz
\]
\end{enumerate}

Therefore option D is correct.
\end{enumerate}

\end{document}
